@Book{NIST:2015:180,
  author =       "{National Institute of Standards and Technology}",
  title =        "{FIPS PUB 180-4}: Secure Hash Standard",
  publisher =    "National Institute for Standards and Technology",
  address =      "Gaithersburg, MD, USA",
  month =        "August",
  year =         "2015",
  bibdate =      "Tues Jan 30 2018",
  note =         "Supersedes FIPS PUB 180-4 2012 March.",
  URL =          "https://csrc.nist.gov/publications/detail/fips/180/4/final",
  abstract =     "This standard specifies hash algorithms that can be used to generate digests of messages. The digests are used to detect whether messages have been changed since the digests were generated. The Applicability Clause of this standard was revised to correspond with the release of FIPS 202, "SHA-3 Standard: Permutation-Based Hash and Extendable-Output Functions," which specifies the SHA-3 family of hash functions, as well as mechanisms for other cryptographic functions to be specified in the future. The revision to the Applicability Clause approves the use of hash functions specified in either FIPS 180-4 or FIPS 202 when a secure hash function is required for the protection of sensitive, unclassified information in Federal applications, including as a component within other cryptographic algorithms and protocols.",
  keywords =     "computer security; cryptography; message digest; hash function; hash algorithm; Federal Information Processing Standards; Secure Hash Standard",
}

@Book{NIST:2013:186,
  author =       "{National Institute of Standards and Technology}",
  title =        "{FIPS PUB 186-4}: Digital Signature Standard (DSS)",
  publisher =    "National Institute for Standards and Technology",
  address =      "Gaithersburg, MD, USA",
  month =        "July",
  year =         "2013",
  bibdate =      "Tues Jan 30 2018",
  note =         "Supersedes FIPS PUB 186-3 2009 June.",
  URL =          "https://csrc.nist.gov/publications/detail/fips/186/4/final",
  abstract =     "The Standard specifies a suite of algorithms that can be used to generate a digital signature. Digital signatures are used to detect unauthorized modifications to data and to authenticate the identity of the signatory. In addition, the recipient of signed data can use a digital signature as evidence in demonstrating to a third party that the signature was, in fact, generated by the claimed signatory. This is known as non-repudiation, since the signatory cannot easily repudiate the signature at a later time. This Standard specifies three techniques for the generation and verification of digital signatures: DSA, ECDSA and RSA. This revision increases the length of the keys allowed for DSA, provides additional requirements for the use of ECDSA and RSA, and includes requirements for obtaining assurances necessary for valid digital signatures.",
  keywords =     "computer security; cryptography; Digital Signature Algorithm; digital signatures; Elliptic Curve Digital Signature Algorithm; Federal Information Processing Standard; public key cryptography",
}
  
@Book{NIST:2013:140,
  author =       "{National Institute of Standards and Technology}",
  title =        "{FIPS PUB 140-2}: Security Requirements for Cryptographic Modules",
  publisher =    "National Institute for Standards and Technology",
  address =      "Gaithersburg, MD, USA",
  day =          "25",
  month =        "May",
  year =         "2001",
  bibdate =      "Tues Jan 30 2018",
  note =         "Supersedes FIPS PUB 140-1 1994 January 11.",
  URL =          "https://csrc.nist.gov/publications/detail/fips/140/2/final",
  abstract =     "This Federal Information Processing Standard (140-2) specifies the security requirements that will be satisfied by a cryptographic module, providing four increasing, qualitative levels intended to cover a wide range of potential applications and environments. The areas covered, related to the secure design and implementation of a cryptographic module, include specification; ports and interfaces; roles, services, and authentication; finite state model; physical security; operational environment; cryptographic key management; electromagnetic interference/electromagnetic compatibility (EMI/EMC); self-tests; design assurance; and mitigation of other attacks.",
  keywords =     "computer security; cryptographic module; FIPS 140-2; validation",
}

@Book{NIST:2013:800107,
  author =       "{National Institute of Standards and Technology}, {Quynh Dang}",
  title =        "{NIST SP 800-107 Revision 1}: Reccomendation for Applications Using Approved Hash Algorithms",
  publisher =    "National Institute for Standards and Technology",
  address =      "Gaithersburg, MD, USA",
  month =        "August",
  year =         "2012",
  bibdate =      "Tues Jan 30 2018",
  note =         "Supersedes FIPS PUB 800-107 2009 February.",
  URL =          "https://csrc.nist.gov/publications/detail/sp/800-107/rev-1/final",
  abstract =     "Hash functions that compute a fixed-length message digest from arbitrary length messages are widely used for many purposes in information security. This document provides security guidelines for achieving the required or desired security strengths when using cryptographic applications that employ the approved hash functions specified in Federal Information Processing Standard (FIPS) 180-4. These include functions such as digital signatures, Keyed-hash Message Authentication Codes (HMACs) and Hash-based Key Derivation Functions (Hash-based KDFs).",
  keywords =     "Digital signatures; hash algorithms; cryptographic hash function; hash function; hash-based key derivation algorithms; hash value; HMAC; message digest; randomized hashing; random number generation; SHA; truncated hash values",
}

@Book{NIST:2013:80090b,
  author =       "{National Institute of Standards and Technology}, {Meltem Sönmez Turan},  {Elaine Barker}, {John Kelsey}, {Kerry McKay}",
  title =        "{NIST SP 800-90b}: Recommendation for the Entropy Sources Used for Random Bit Generation",
  publisher =    "National Institute for Standards and Technology",
  address =      "Gaithersburg, MD, USA",
  month =        "January",
  year =         "2016",
  bibdate =      "Tues Jan 30 2018",
  URL =          "",
  abstract =     "This Recommendation specifies the design principles and requirements for the entropy sources used by Random Bit Generators, and the tests for the validation of entropy sources. These entropy sources are intended to be combined with Deterministic Random Bit Generator mechanisms that are specified in SP 800-90A to construct Random Bit Generators, as specified in SP 800-90C.",
  keywords =     "Conditioning functions; Entropy source; health testing; IID testing; min-entropy; noise source; predictors; random number generators",
}

@Book{NIST:2013:80090c,
  author =       "{National Institute of Standards and Technology}, {Elaine Barker}, {John Kelsey}",
  title =        "{NIST SP 800-90c}: Recommendation for Random Bit Generator (RBG) Constructions",
  publisher =    "National Institute for Standards and Technology",
  address =      "Gaithersburg, MD, USA",
  month =        "April",
  year =         "2016",
  bibdate =      "Tues Jan 30 2018",
  URL =          "https://csrc.nist.gov/publications/detail/sp/800-90c/draft",
  abstract =     "This Recommendation specifies constructions for the implementation of random bit generators (RBGs). An RBG may be a deterministic random bit generator (DRBG) or a non-deterministic random bit generator (NRBG). The constructed RBGs consist of DRBG mechanisms, as specified in NIST Special Publication (SP) 800-90A, and entropy sources, as specified in SP 800-90B.",
  keywords =     "deterministic random bit generator (DRBG); entropy; entropy source; non-deterministic random bit generator (NRBG); random number generator; construction; randomness source",
}

@Book{NIST:2013:80090a,
  author =       "{National Institute of Standards and Technology}, {Elaine Barker}, {John Kelsey}",
  title =        "{NIST SP 800-90a}: Recommendation for Random Bit Generator (RBG) Constructions",
  publisher =    "National Institute for Standards and Technology",
  address =      "Gaithersburg, MD, USA",
  month =        "June",
  year =         "2015",
  bibdate =      "Tues Jan 30 2018",
  URL =          "http://nvlpubs.nist.gov/nistpubs/SpecialPublications/NIST.SP.800-90Ar1.pdf",
  abstract =     "This Recommendation specifies mechanisms for the generation of random bits using deterministic methods. The methods provided are based on either hash functions 
  keywords =     "Deterministic random bit generator (DRBG); entropy; hash function; random number generator",
}

@Book{NIST:2013:80057,
  author =       "{National Institute of Standards and Technology}, {Elaine Barker}",
  title =        "{NIST SP 800-57 Part 1 Revision 4}: Recommendation for Key Management",
  publisher =    "National Institute for Standards and Technology",
  address =      "Gaithersburg, MD, USA",
  month =        "January",
  year =         "2016",
  bibdate =      "Tues Jan 30 2018",
  URL =          "https://csrc.nist.gov/publications/detail/sp/800-57-part-1/rev-4/final",
  abstract =     "This Recommendation provides cryptographic key management guidance. It consists of three parts. Part 1 provides general guidance and best practices for the management of cryptographic keying material. Part 2 provides guidance on policy and security planning requirements for U.S. government agencies. Finally, Part 3 provides guidance when using the cryptographic features of current systems.",
  keywords =     "archive; assurances; authentication; authorization; availability; backup; compromise; confidentiality; cryptanalysis; cryptographic key; cryptographic module; digital signature; hash function; key agreement; key management; key management policy; key recovery; key transport; originator-usage period; private key; public key; recipient-usage period; secret key; split knowledge; trust anchor",
}